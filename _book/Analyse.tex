\documentclass[]{book}
\usepackage{lmodern}
\usepackage{amssymb,amsmath}
\usepackage{ifxetex,ifluatex}
\usepackage{fixltx2e} % provides \textsubscript
\ifnum 0\ifxetex 1\fi\ifluatex 1\fi=0 % if pdftex
  \usepackage[T1]{fontenc}
  \usepackage[utf8]{inputenc}
\else % if luatex or xelatex
  \ifxetex
    \usepackage{mathspec}
  \else
    \usepackage{fontspec}
  \fi
  \defaultfontfeatures{Ligatures=TeX,Scale=MatchLowercase}
\fi
% use upquote if available, for straight quotes in verbatim environments
\IfFileExists{upquote.sty}{\usepackage{upquote}}{}
% use microtype if available
\IfFileExists{microtype.sty}{%
\usepackage{microtype}
\UseMicrotypeSet[protrusion]{basicmath} % disable protrusion for tt fonts
}{}
\usepackage[margin=1in]{geometry}
\usepackage{hyperref}
\hypersetup{unicode=true,
            pdftitle={Cours d'Analyse},
            pdfauthor={Franck Corset},
            pdfborder={0 0 0},
            breaklinks=true}
\urlstyle{same}  % don't use monospace font for urls
\usepackage{natbib}
\bibliographystyle{apalike}
\usepackage{longtable,booktabs}
\usepackage{graphicx,grffile}
\makeatletter
\def\maxwidth{\ifdim\Gin@nat@width>\linewidth\linewidth\else\Gin@nat@width\fi}
\def\maxheight{\ifdim\Gin@nat@height>\textheight\textheight\else\Gin@nat@height\fi}
\makeatother
% Scale images if necessary, so that they will not overflow the page
% margins by default, and it is still possible to overwrite the defaults
% using explicit options in \includegraphics[width, height, ...]{}
\setkeys{Gin}{width=\maxwidth,height=\maxheight,keepaspectratio}
\IfFileExists{parskip.sty}{%
\usepackage{parskip}
}{% else
\setlength{\parindent}{0pt}
\setlength{\parskip}{6pt plus 2pt minus 1pt}
}
\setlength{\emergencystretch}{3em}  % prevent overfull lines
\providecommand{\tightlist}{%
  \setlength{\itemsep}{0pt}\setlength{\parskip}{0pt}}
\setcounter{secnumdepth}{5}
% Redefines (sub)paragraphs to behave more like sections
\ifx\paragraph\undefined\else
\let\oldparagraph\paragraph
\renewcommand{\paragraph}[1]{\oldparagraph{#1}\mbox{}}
\fi
\ifx\subparagraph\undefined\else
\let\oldsubparagraph\subparagraph
\renewcommand{\subparagraph}[1]{\oldsubparagraph{#1}\mbox{}}
\fi

%%% Use protect on footnotes to avoid problems with footnotes in titles
\let\rmarkdownfootnote\footnote%
\def\footnote{\protect\rmarkdownfootnote}

%%% Change title format to be more compact
\usepackage{titling}

% Create subtitle command for use in maketitle
\newcommand{\subtitle}[1]{
  \posttitle{
    \begin{center}\large#1\end{center}
    }
}

\setlength{\droptitle}{-2em}

  \title{Cours d'Analyse}
    \pretitle{\vspace{\droptitle}\centering\huge}
  \posttitle{\par}
    \author{Franck Corset}
    \preauthor{\centering\large\emph}
  \postauthor{\par}
      \predate{\centering\large\emph}
  \postdate{\par}
    \date{2019-01-21}

\usepackage{booktabs}

\usepackage{amsthm}
\newtheorem{theorem}{Theorem}[chapter]
\newtheorem{lemma}{Lemma}[chapter]
\newtheorem{corollary}{Corollary}[chapter]
\newtheorem{proposition}{Proposition}[chapter]
\newtheorem{conjecture}{Conjecture}[chapter]
\theoremstyle{definition}
\newtheorem{definition}{Definition}[chapter]
\theoremstyle{definition}
\newtheorem{example}{Example}[chapter]
\theoremstyle{definition}
\newtheorem{exercise}{Exercise}[chapter]
\theoremstyle{remark}
\newtheorem*{remark}{Remark}
\newtheorem*{solution}{Solution}
\let\BeginKnitrBlock\begin \let\EndKnitrBlock\end
\begin{document}
\maketitle

{
\setcounter{tocdepth}{1}
\tableofcontents
}
\hypertarget{preambule}{%
\chapter{Préambule}\label{preambule}}

Vous trouverez des vidéos et des tests dans la page moodle développée par Agnes Hamon à l'adresse suivante :

\url{https://cours.univ-grenoble-alpes.fr/course/view.php?id=3562\#section-0}

\hypertarget{intro}{%
\chapter{Introduction}\label{intro}}

Le plan de ce cours est le suivant :

\begin{itemize}
\item
  Rappels sur la continuité et la dérivabilité
\item
  Primitives et Intégration
\end{itemize}

\hypertarget{continuite-et-derivabilite}{%
\chapter{Continuité et dérivabilité}\label{continuite-et-derivabilite}}

\hypertarget{introduction-et-rappels}{%
\section{Introduction et rappels}\label{introduction-et-rappels}}

\BeginKnitrBlock{definition}[Application]
\protect\hypertarget{def:unnamed-chunk-1}{}{\label{def:unnamed-chunk-1} \iffalse (Application) \fi{} }Soient \(E\) et \(F\), deux ensembles et \(f\) est une application de \(E\) dans \(F\) si et seulement si pour tout \(x\in E\), on associe un \textbf{unique} élément \(y\in F\). \(y=f(x)\) est appelé image de \(x\) par \(f\) et \(x\) est appelé antécédent de \(y\) par \(f\).
\EndKnitrBlock{definition}

Remarque : C'est la définition d'une application et non pas d'une fonction.

\BeginKnitrBlock{definition}[Application injective]
\protect\hypertarget{def:unnamed-chunk-2}{}{\label{def:unnamed-chunk-2} \iffalse (Application injective) \fi{} }Soient \(E\) et \(F\), deux ensembles et \(f\) est une application de \(E\) dans \(F\). \(f\) est \textbf{injective} si et seulement si tout élément \(y\) de \(F\) admet \textbf{au plus} un antécédent dans \(E\)
\EndKnitrBlock{definition}

\BeginKnitrBlock{definition}[Application surjective]
\protect\hypertarget{def:unnamed-chunk-3}{}{\label{def:unnamed-chunk-3} \iffalse (Application surjective) \fi{} }Soient \(E\) et \(F\), deux ensembles et \(f\) est une application de \(E\) dans \(F\). \(f\) est \textbf{surjective} si et seulement si tout élément \(y\) de \(F\) admet \textbf{au moins} un antécédent dans \(E\)
\EndKnitrBlock{definition}

\BeginKnitrBlock{definition}[Application bijective]
\protect\hypertarget{def:unnamed-chunk-4}{}{\label{def:unnamed-chunk-4} \iffalse (Application bijective) \fi{} }Soient \(E\) et \(F\), deux ensembles et \(f\) est une application de \(E\) dans \(F\). \(f\) est \textbf{bijective} si et seulement si \(f\) est injective \textbf{et} surjective. Autrement dit, \(f\) est \textbf{bijective} si et seulement si tout élément \(y\) de \(F\) admet \textbf{un et un seul} antécédent \(x\) dans \(E\):
\[
\forall y \in F, \exists ! x\in E,\, y=f(x)  
\]
\EndKnitrBlock{definition}

\BeginKnitrBlock{definition}[Application bijection réciproque]
\protect\hypertarget{def:unnamed-chunk-5}{}{\label{def:unnamed-chunk-5} \iffalse (Application bijection réciproque) \fi{} }Soient \(E\) et \(F\), deux ensembles et \(f\) est une application bijective de \(E\) dans \(F\). On définit l'application, notée \(f^{-1}\), qui à chaque \(y\) de \(F\) associe l'unique élément \(x\) de \(E\), tel que \(y=f(x)\). On note \(x=f^{-1}(y)\). Cette application est appelée bijection réciproque de \(f\).
\EndKnitrBlock{definition}

Pour toute la suite, nous nous intéressons aux fonctions réelles, c'est à dire où les ensembles de départ et d'arrivée sont des sous-ensemble de \(\mathbb R\).

\BeginKnitrBlock{definition}[Continuité]
\protect\hypertarget{def:unnamed-chunk-6}{}{\label{def:unnamed-chunk-6} \iffalse (Continuité) \fi{} }Soit \(f\) une fonction définie sur un intervalle \(I=]a,b[\) (\(a<b\)), ouvert de \(\mathbb R\), à valeurs réelles. Soit \(x_0\in I\), on dit que \(f\) est continue en \(x_0\) si et seulement si
\[
\lim_{x\rightarrow x_0^-}f(x)=\lim_{x\rightarrow x_0^+}f(x)=f(x_0).
\]
On dit que \(f\) est une fonction \textbf{continue} sur \(]a;b[\) si elle est \{\bf continue\} en tout point \(x_0\) de \(]a;b[\).
\EndKnitrBlock{definition}

\hypertarget{methods}{%
\chapter{Methods}\label{methods}}

We describe our methods in this chapter.

\hypertarget{applications}{%
\chapter{Applications}\label{applications}}

Some \emph{significant} applications are demonstrated in this chapter.

\hypertarget{example-one}{%
\section{Example one}\label{example-one}}

\hypertarget{example-two}{%
\section{Example two}\label{example-two}}

\hypertarget{final-words}{%
\chapter{Final Words}\label{final-words}}

We have finished a nice book.

\bibliography{book.bib,packages.bib}


\end{document}
